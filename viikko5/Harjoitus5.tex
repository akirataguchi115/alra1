	\documentclass[11pt,a4paper]{amsart}

\RequirePackage[hyphens]{url}

\RequirePackage[OT1]{fontenc}
\RequirePackage{amsthm,amsmath}
\RequirePackage[numbers]{natbib}
\RequirePackage[colorlinks,citecolor=blue,urlcolor=blue]{hyperref}
\RequirePackage{cleveref}

\usepackage{graphicx}
\usepackage{tikz}
\usepackage{enumitem}
\usepackage[utf8]{inputenc}
\usepackage[english]{babel}
\usepackage{latexsym}
\usepackage{amssymb}
\usepackage{amsmath}
\usepackage{pgfplots}
\usepackage{tikz}
\usepackage{graphicx}
\usepackage[most]{tcolorbox}
\usepackage{nameref}
\usepackage{enumitem}
\usepackage{randomwalk}
\usepackage{geometry}


\usetikzlibrary{arrows,calc}


\usetikzlibrary{automata, positioning}


\tcbuselibrary{theorems}



\newtheorem{theorem}{Theorem}
\newtheorem{lemma}[theorem]{Lemma}
\newtheorem*{lemma*}{Lemma}
\newtheorem{proposition}[theorem]{Proposition}
\newtheorem{corollary}[theorem]{Corollary}
\newtheorem{conjecture}[theorem]{Conjecture}
\newtheorem{definition}[theorem]{Definition}
\newtheorem{remark}[theorem]{Remark}
\newtheorem{assumption}[theorem]{Assumption}
\newtheorem{claim}[theorem]{Claim}
\newtheorem{fact}[theorem]{Fact}
\newtheorem*{fact*}{Fact}

\tcbuselibrary{theorems}
\newtcbtheorem[number within=section]{Definitions}{Määritelmä}{breakable,code={\edef\@currentlabelname{#2}},colframe=green!45!black,fonttitle=\bfseries}{Definitions}
\newtcbtheorem[number within=section]{Examples}{Esimerkki}{breakable,code={\edef\@currentlabelname{#2}},colframe=orange!85!black,fonttitle=\bfseries}{Examples}
\newtcbtheorem[number within=section]{Theorems}{Lause}{breakable,code={\edef\@currentlabelname{#2}},colframe=blue!85!black,fonttitle=\bfseries}{Theorems}
\newtcbtheorem[number within=section]{Propositions}{Propositio}{breakable,code={\edef\@currentlabelname{#2}},colframe=teal!85!black,fonttitle=\bfseries}{Propositions}
\newtcbtheorem[number within=section]{Lemmas}{Lemma}{breakable,code={\edef\@currentlabelname{#2}},colframe=cyan!85!black,fonttitle=\bfseries}{Lemmas}
\newtcbtheorem{Comments}{Huomautus}{breakable,code={\edef\@currentlabelname{#2}},colframe=pink!85!black,fonttitle=\bfseries}{Comments}
\newtcbtheorem{Exercises}{Harjoitustehtävä}{breakable,code={\edef\@currentlabelname{#2}},colframe=yellow!55!black,fonttitle=\bfseries}{Exercises}
\newtcbtheorem[number within=section]{Warnings}{Varoitus}{breakable,code={\edef\@currentlabelname{#2}},colframe=red!65!black,fonttitle=\bfseries}{Warnings}
\newtcolorbox{Proof}{breakable,colframe=gray!55!black,fonttitle=\bfseries,title=Todistus:}
\newtcolorbox{Solution}{breakable,colframe=gray!55!black,fonttitle=\bfseries,title=Ratkaisu:}
\newtcbtheorem[number within=section]{HExercises}{Teoreettinen lisätehtävä}{breakable,code={\edef\@currentlabelname{#2}},colframe=yellow!35!black,fonttitle=\bfseries}{HExercises}

\newtcbtheorem[number within=section]{CExercises}{Laskennallinen lisätehtävä}{breakable,code={\edef\@currentlabelname{#2}},colframe=yellow!75!black,fonttitle=\bfseries}{HExercises}


\usetikzlibrary{decorations.markings}

% axis style, ticks, etc
\pgfplotsset{every axis/.append style={
                    axis x line=middle,    % put the x axis in the middle
                    axis y line=middle,    % put the y axis in the middle
                    axis line style={->}, % arrows on the axis
                    ymajorticks=false,
                    %xlabel={$x$},          % default put x on x-axis
                    %ylabel={$y$},          % default put y on y-axis
                    xtick={-1,1},                     
                    }}
% arrows as stealth fighters
\tikzset{>=stealth}


\newcommand{\T}{\mathcal{T}}
\newcommand{\N}{\mathbb{N}}
\newcommand{\Z}{\mathbb{Z}}
\newcommand{\R}{\mathbb{R}}
\newcommand{\C}{\mathbb{C}}
\newcommand{\E}{\mathbb{E}}
\newcommand{\Q}{\mathbb{Q}}
\newcommand{\F}{\mathcal{F}}
\newcommand{\Tr}{\mathrm{Tr}}
\newcommand{\M}{\mathrm{Markov}}
\newcommand{\1}{\mathbf{1}}
\newcommand{\red}{\color{red}}
\newcommand{\Exp}{\mathrm{Exp}}
\newcommand{\Poi}{\mathrm{Poi}}
\newcommand{\Unif}{\mathrm{Unif}}

\renewcommand{\P}{\mathbb P}
\renewcommand{\S}{\mathcal S}
\renewcommand{\Re}{\mathrm{Re} \,}
\renewcommand{\Im}{\mathrm{Im} \,}
\renewcommand{\epsilon}{\varepsilon}


%\numberwithin{equation}{section}
%\numberwithin{theorem}{section}
%\numberwithin{figure}{section}

\usepackage{eurosym}


\usepackage{fullpage}
%\usepackage[notref, notcite]{showkeys} %Comment to hide all the labels in the file


\addto\captionsenglish{% Replace "english" with the language you use
  \renewcommand{\contentsname}%
    {Sisällysluettelo}%
}

\addto\captionsenglish{% Replace "english" with the language you use
  \renewcommand{\figurename}%
    {Kuva}%
}

\addto\captionsenglish{% Replace "english" with the language you use
  \renewcommand{\abstractname}%
    {Abstrakti}%
}

\addto\captionsenglish{% Replace "english" with the language you use
  \renewcommand{\refname}%
    {Viitteet}%
}

\addto\captionsenglish{% Replace "english" with the language you use
  \renewcommand{\appendixname}%
    {Liite}%
}

\usepackage{tikz-cd}


	
\setcounter{tocdepth}{1}

\title{Algebralliset rakenteet I 2022 -- Viikon 5 harjoitustehtävät}

\author{Christian Webb}

\email{christian.webb@helsinki.fi}
\begin{document}
\maketitle

\medskip

\begin{Exercises}{Merkinnät ja peruskäsitteet}{not}
Pitävätkö seuraavat väitteet paikkansa? Ei tarvitse perustella. 
\begin{enumerate}
\item $\Z=\langle -1\rangle$.
\item Jos $(G,\star)$ on ryhmä ja $g\in G$, niin $\langle g^2\rangle \subset \langle g\rangle$.
\item Jos $(G,\star)$ on ryhmä ja $g\in G$, niin $\langle g^{-1}\rangle =\langle g\rangle$.
\item Syklin $(12)\in S_3$ kertaluku on $3$.
\item Kaikki sykliset ryhmät ovat keskenään isomorfisia.
\end{enumerate}
\end{Exercises}

\begin{Solution}
	\begin{enumerate}
		\item \textcolor{red}{Ei päde}
		\item Pätee
		\item \textcolor{red}{Ei päde}
		\item Pätee
		\item Pätee
	\end{enumerate}
\end{Solution}


\begin{Exercises}{Transpositiot ja symmetrisen ryhmän virittäminen}{transp}
Permutaatiota $\sigma \in S_n$ kutsutaan \emph{transpositioksi}, jos se on $2$-sykli: $\sigma=(ab)$ jollakin $a,b\in \{1,...,n\}: a\neq b$.
\begin{enumerate}
\item Osoita, että mikä tahansa sykli $(a_1a_2...a_k)$ voidaan kirjoittaa transpositioiden tulona:
\[
(a_1a_2....a_k)=(a_1a_k)(a_1a_{k-1})\cdots (a_1a_2).
\]
\item Olkoon $T_n=\{\sigma\in S_n: \sigma \text{ on transpositio}\}$ transpositioiden joukko. Osoita, että $\langle T_n\rangle=S_n$.
\end{enumerate}
{\bf Vihje:} mieti ensimmäisessä kohdassa mille alkiolle kyseinen transpositioiden tulo kuvaa alkion $a_j$. Jälkimmäisessä kohdassa on varmaankin hyvä ajatus muistaa, että jokainen permutaatio voidaan kirjoittaa syklien tulona, ja käyttää ensimmäistä osaa.
\end{Exercises}

\begin{Exercises}{Kertaluku}{order}
\begin{enumerate}
\item Olkoon $\tau \in S_n$ $k$-sykli: $\tau=(a_1...a_k)$. Mikä on alkion $\tau$ kertaluku? (Perustele)
\item Olkoon $\tau_1,\tau_2\in S_n$ erillisiä syklejä: $\tau_1=(a_1\cdots a_k)$ ja $\tau_2=(b_1...b_l)$ missä $a_i\neq b_j$. Mikä on alkion $\tau_1\tau_2$ kertaluku? (Perustele)
\item Osaatko kuvata mielivaltaisen permutaation $\sigma\in S_n$ kertalukua sen syklihajotelman avulla?
\end{enumerate}
{\bf Vihje:} jos ensimmäisessä kohdassa et heti pääse vauhtiiin, voit miettiä mikä on 2-syklin kertaluku, mikä on 3-syklin kertaluku, ja miettiä, että osaatko yleistää. 2-kohtaan voisi riittää vihjeeksi lyhenne pyj. 3-kohtaan mainittakoon, että pyj voidaan määritellä myös useammalle luvulle.
\end{Exercises}

\begin{Exercises}{Jäännösluokkaryhmän virittäminen}{cong}
Olkoon $p$ jokin alkuluku ja tarkastellaan jäännösluokkaryhmää $\Z_p$. Osoita, että $\langle g\rangle=\Z_p$ jokaisella $g\in \Z_p\setminus \{[0]_p\}$.

{\bf Vihje:} Luentomuistiinpanojen Propositiosta 11.3 saattaa olla tässä hyötyä. Erityisesti kannattaa muistaa, että jos $k\in \{1,..,p-1\}$ ja $p$ on alkuluku, niin $\mathrm{syt}(k,p)=1$. Tehtävä on hyvin lyhyt kun löydät sopivan lähtökohdan.
\end{Exercises}

\begin{Solution}
  Olkoon $p$ jokin alkuluku. Proposition 11.3.(5) mukaan
  \begin{align*}
    \text{Jos } syt(k,n)=1\text{, niin} \langle g^k \rangle=G
  \end{align*}
  Koska tiedämme, että
  \begin{align*}
    \text{jos } k\in \{1,\cdot,p-1\} \text{ ja } p \text{ on alkuluku, niin } syt(k,p)=1
  \end{align*}
  Siis $\langle g^k \rangle = G$.

  Proposition 11.3.(2) mukaan kaikilla $k\in\mathbb{Z}$ pätee $\langle g^k \rangle=\langle g^{syt(n,k)} \rangle$. Aikaisemman perusteella $\langle g^{syt(n,k)} \rangle=\langle g^1 \rangle$. Koska $\mathbb{Z}_p$ on $1$:n virittämä ryhmä, voimme todeta $\langle g \rangle =\mathbb{Z}_p \text{, jokaisella } g \in \mathbb{Z}_p \setminus \{[0]_p\}$.
\end{Solution}

\begin{Exercises}{Syklisyyden puute}{noncycl}
\begin{enumerate}
\item Osoita, että $(\Q,+)$ ei ole syklinen.
\item Osoita, että $S_4$:n aliryhmä $H=\{e,(12),(34),(12)(34)\}$ ei ole syklinen (saat tietää, että $H$ on ryhmä).
\end{enumerate}
{\bf Vihje: } Muistiinpanojen Propositio 11.1 ja 2. viikon harjoitukset saattavat auttaa ensimmäisessä kohdassa. Toisessa, huomaa, että Propositio 11.1 sanoisi, että $H$:n tulisi olla isomorfinen $\Z_4$:n kanssa. Koita miettiä miksi tämä on mahdotonta (huom, jokainen $H$:n alkio $\sigma$ toteuttaa $\sigma^2=e$).
\end{Exercises}

\begin{Solution}
  \begin{enumerate}
    \item  Tehdään ristiriitatodistus. Oletetaan, että $(\mathbb{Q},+)$ on syklinen. Proposition 11.1 jompi kumpi ehdoista täytyy päteä, jotta $(\mathbb{Q},\star)$ on syklinen ryhmä. Koska $\mathbb{Q}$:n kertaluku ei ole äärellinen, voimme keskittyä tutkimaan ehtoa $\mathbb{Q} \cong \mathbb{Z}$. Harjoituksen 2.6 perusteella huomaamme, ettei viimeinenkään ehto pidä paikkaansa. Siis $(\mathbb{Q},+)$ ei ole syklinen.
    \item Huomaamme propositiosta 11.1 bijektioristiriidan kohdassa (2), siis $H \not \cong \mathbb{Z}_4$. Näin ollen $H$ ei ole syklinen.
  \end{enumerate}
\end{Solution}

\begin{Exercises}{Virittämisestä}{gen}
Anna alla oleville ryhmille $(G,\star)$ esimerkki joukosta $S\subset G$, jolla $S\neq G$ (ja sovitaan myös, että $S\neq G\setminus \{e\}$), mutta $\langle S\rangle=G$. Ei tarvitse perustella.
\begin{enumerate}
\item $(G,\star)=(\R,+)$.
\item $(G,\star)=(\{-1,1\}\times \{-1,1\},\cdot)$, missä $(a,b)\cdot (x,y)=(ax,by)$, missä $ax$ ja $by$ ovat kokonaislukujen tuloja.
\item $(G,\star)=(0,\infty)$ varustettu tulolla.
\end{enumerate}
\end{Exercises}
\end{document}